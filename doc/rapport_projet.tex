\documentclass[15pt, a4paper]{article}

\usepackage[utf8]{inputenc}
\usepackage[francais]{babel}
\usepackage{verbatim}
\usepackage{graphicx}
\usepackage{listings}

\title{Licence 3 Informatique 2009-2010\\
Projet d'Algorithmique I\\
``Implantation d'un dictionnaire de type T9''}

\author{Châtel Grégory, gchatel\\Bassinot Hervé, hbassino}

\begin{document}

\maketitle{}

\clearpage

\tableofcontents

\clearpage

\abstract{Pour faciliter la saisie des chaînes de caractères, certains téléphones portables utilisent
des dictionnaires de prédiction, comme par exemple le dictionnaire T9. La méthode consiste
à saisir chaque mot en tapant une fois le chiffre associé à chaque lettre du mot, comme si on
tapait le mot sur un clavier normal. Par exemple, pour saisir “jour”, il faut taper “5687”. A
partir de cette suite de chiffres le téléphone, aidé d’un dictionnaire contenant tous les mots
du français, trouve le mot qu’on voulait saisir. Dans la plupart des cas, il ne correspond qu’un
seul mot à la suite de chiffres, mais parfois plusieurs choix peuvent être proposés : une touche
(souvent 0) permet alors de choisir parmi les solutions. Par exemple, les mots “jour”, “loup”,
“lots” correspondent à la suite “5687”. Le but du projet est d’implanter le logiciel proposant
les mots se trouvant dans le téléphone portable.}

\clearpage

\section{Présentation de l'application}

Le but de l'application ``Implantation d'un dictionnaire de type T9'' est de permettre de construire 
un dictionnaire T9 utilisable dans notre application pour chercher des mots à partir d'un fichier texte
faisant office de dictionnaire. Le scénario du programme est simple, l'utilisateur lance le programme 
avec comme argument de la ligne de commande un fichier représentant le dictionnaire à utiliser. Le programme 
va alors construire l'arborescence correspondant aux mots du fichier. Le programme permet ensuite à 
l'utilisateur de chercher des mots correspondant à une entrée formée d'une suite de numero de touche,
c'est à dire les caractères numériques 2 à 9. Une fois le mot saisi le programme va parcourir le 
dictionnaire qu'il possède en mémoire et afficher les résultats correspondants à l'entrée de l'utilisateur
couplée au options du programme (affichage préfixe, \dots).

\bigskip

Le fichier texte transmis au programme devra être écrit sous une forme spécifique. Il ne doit contenir
que des caractères alphabétiques non accentués, à raison d'un mot par ligne.

\bigskip

Diverses options seront proposées à l'utilisateur, en effet, il est possible d'effectuer 
les actions suivantes :

\bigskip

\begin{itemize}
\item L'affichage de l'ensemble des mots correspondants à l'entrée saisie au clavier.
\item L'affichage de l'ensemble des mots ayant un préfixe correspondant à l'entrée saisie au clavier.
\item L'affichage complet des statistiques du dictionnaire (Nombre de mots connus par le programme,
  nombre de noeuds des arbres du dictionnaire).
\item L'enregistrement du dictionnaire dans l'ordre lexicographique dans un fichier.
\item L'enregistrement du dictionnaire dans un fichier sous la fome d'un graphe.
\item La minimisation de l'arbre du dictionnaire.
\end{itemize}

\bigskip

\noindent Toutes ces fonctionnalités seront détaillées plus précisement dans la suite du rapport.
Par défaut, le programme affiche les mots qui correspondent strictement à l'entrée de l'utilisateur, sans
minimiser ses arbres.

\bigskip

Durant la réalisation de ce projet, un point d'honneur a été mis à écrire du code source modulaire pour en
faciliter la mise à jour, la correction et la lisibilité.

\section{Notice d'utilisation du programme}

\noindent Cette partie du rapport explique en détail comment utiliser le programme ainsi l'ensemble de ses 
options.

\subsection{Compilation du programme T9}
Pour permettre la compilation du programme, nous avons écrit un makefile permetant entre autre de compiler 
l'ensemble des fichiers du projet. Pour construire l'exécutable T9, il faut se placer dans le répertoire 
principal du projet et taper la commande suivante :

\begin{verbatim}
make T9
\end{verbatim}

ou bien plus simplement

\begin{verbatim}
make
\end{verbatim}

Cette commande va donc compiler l'ensemble des modules du programme avec les options suivantes :  
-ansi -Wall -pedantic -Werror -O3. Cette combinaison de flag permet d'assurer le respect de la norme ansi et
pedantic du langage C ainsi que la compilation du programme sans aucun warning. De plus, l'option -O3 indique
à gcc qu'il doit effectuer des optimisations affectant le temps d'exécution du programme.

L'ensemble des fichiers objets générés à la compilation sont stockés dans le répértoire obj. L'éxecutable est créé 
dans le répertoire principal du projet sous le nom de ``T9''.

\bigskip

Notre fichier makefile permet également de supprimer l'ensemble des fichiers objets générés lors de la
compilation grace à la commande suivante :

\begin{verbatim}
make clean
\end{verbatim}

\noindent Nous pouvons également à l'aide du Makefile créer une archive du projet à l'aide de la commande :

\begin{verbatim}
make archive
\end{verbatim}

Cette commande supprime tous les fichiers objet puis creer une archive complète du projet, il est possible
de changer le nom de cette archive simplement dans le fichier makefile.

\subsection{Execution du programme T9}

L'ensemble du jeu d'essais du programme est situé dans le repertoire corpus. Le programme ne charge pas 
de dictionnaire par défaut, il est donc necéssaire de lui passer le nom d'un dictionnaire à chaque lancement.

\subsubsection{Execution normale}

Pour exécuter le programme avec son comportement par défaut, on utilise la ligne de commande suivante : 

\begin{verbatim}
./T9 corpus/dela-fr-public.dic.simple.txt
\end{verbatim}

Ainsi cette commande va initialiser le dictionnaire du programme avec le fichier 
dela-fr-public.dic.simple.txt situé dans le répertoire corpus. 

\subsubsection{Execution avec options}

\noindent Les options supportées par le programme sont les suivantes :

\begin{itemize}
\item -t : L'enregistrement du dictionnaire dans l'ordre lexicographique dans un fichier externe.
\item -n : L'affichage des statistiques du dictionnaire.
\item -b : L'enregistrement du dictionnaire dans un fichier sous forme de graphe.
\item -x : L'affichage de l'ensemble des mots ayant un préfixe correspondant à l'entrée saisie.
\item -m : La minimisation des arbres du programme.
\end{itemize}

\bigskip

\noindent Pour lancer le programme avec une option on utilise la ligne suivante :

\begin{verbatim}
./T9 corpus/dela-fr-public.dic.simple.txt -option
\end{verbatim}

\noindent Où -option est le nom de l'option à appliquer au programme.

\bigskip

\noindent Exemple : Cette commande lance le programme T9 avec l'option d'affichage prefixe.

\begin{verbatim}
./T9 corpus/dela-fr-public.dic.simple.txt -x
\end{verbatim}

\bigskip

Il est également possible de cumuler les options avec la syntaxe suivante :
\begin{verbatim}
./T9 corpus/dela-fr-public.dic.simple.txt -xbm
\end{verbatim}
Ceci lance le programme avec les options -x -b et -m du programme. Le fait que les options de 
la ligne de commande soient traitées en utilisant getopt autorise une grande souplesse dans
le passage des arguments au programme.

\subsubsection{Gestion des erreurs d'exécution}

Si une option passée par la ligne de commande au programme n'est pas valide, l'éxecution prend fin 
immediatement. De même, si plusieurs fichiers dictionnaires sont passés en argument ou bien si le fichier
représentant le dictionnaire ne peut pas être ouvert, l'exécution se termine. En cas d'erreur, les messages
suivant sont affichés :

\begin{verbatim}
Erreur : Argument transmis incorrect.
\end{verbatim}

\noindent Indique que les options transmises sont incorrectes ou que plusieurs noms de dictionnaire sont présents.

\begin{verbatim}
Erreur : Nom de fichier invalide.
\end{verbatim}

\noindent Indique que le dictionnaire ne peut pas être ouvert.

\subsection{Déroulement du programme T9}

Durant le lancement du programme, les arbres représentant le dictionnaire sont construit et les éventuelles options
sont appliquées. Durant la phase de construction des arbres, le programme récupère un à un les mots du dictionnaire puis 
les insère dans les arbres. Notre programme ne prend en compte que les caractères alphabétiques minuscules non accentués.
Si certains mots du dictionnaire ne sont pas valides, le programme les passes, un message d'erreur est cependant affiché 
à l'utilisateur. Lorsque le programme a terminé son initialisation et qu'il est pret à fonctionner, il l'indique à 
l'utilisateur grâce au message suivant :

\begin{verbatim}
Saisi OK
\end{verbatim}

L'utilisateur peut maintenant entrer les mots qu'il souhaite rechercher sous la forme d'une suite de caractère de
2 à 9 puis valider en appuyant sur la touche entrée.

\bigskip

\noindent Si les caractères entrés ne sont pas valides, le programme affiche le message d'erreur suivant :

\begin{verbatim}
Mot non valide !!
\end{verbatim}

\noindent Si aucun mot ne correspond à la chaîne entrée par l'utilisateur, le programme affiche le message suivant :

\begin{verbatim}
Aucun mot trouve !!
\end{verbatim}

\noindent Dans le cas contraire le programme affiche l'ensemble des mots trouvés à raison d'un mot par ligne.

\section{Structures de données}

\subsection{Stockage du dictionnaire}

\noindent Pour stocker les mots du dictionnaires, nous avons choisi d'utiliser les structures suivantes : 

\begin{lstlisting}[language=c]
#define NB_LETTRE 4
#define NB_TOUCHE 8

typedef enum statut {
    TERMINAL,
    NON_TERMINAL
} Statut;

typedef enum visite {
    VISITE,
    NON_VISITE
} Visite;

typedef struct noeudAuxiliaire {
    struct noeudAuxiliaire * fils[NB_LETTRE];
    Statut statut;
    Visite visite;
    unsigned int hache;
    unsigned int numero;
} NoeudAuxiliaire, * ArbreAuxiliaire;

typedef struct noeudPrincipal {
    struct noeudPrincipal * touche[NB_TOUCHE];
    ArbreAuxiliaire arbreAux;
    Visite visite;
    unsigned int hache;
    unsigned int numero;
} NoeudPrincipal, * ArbrePrincipal;
\end{lstlisting}

Nous avons choisi une structure d'arbre lexicographique utilisant un tableau de fils. Cette méthode a pour 
avantage d'être très simple à manipuler contrairement à une représentation par fils gauche frère droit
qui fait intervenir un parcours de liste et qui utilise plus de place dans le cas de noeuds à fort facteur
de branchement.

\bigskip

\noindent Dans la structure de l'arbre auxiliaire, on trouve : 

\begin{itemize}
\item Un tableau de pointeurs sur noeud auxiliaire représentant les fils de ce noeud.
\item Un champs statut qui indique si le noeud est terminal ou non.
\item Un champs visite qui indique si le champs a déjà été visité lors du calcul du nombre de noeuds 
  auxiliaires (ce champs est rendu nécessaire par la fusion et la minimisation, il permet de ne pas 
  compter plusieurs fois le même noeud).
\item Un champs hache qui contiendra la valeur du haché du noeud lors de la minimisation de l'arbre.
\item Un champs numéro qui contiendra le numéro du noeud lors de la minimisation (ce fonctionnement sera
  expliqué plus loin dans le rapport).
\end{itemize}

\bigskip

\noindent Dans la structure de l'arbre principal, on trouve :

\begin{itemize}
\item Un tableau de pointeurs sur noeud principal représentant les fils de ce noeud.
\item Un champs arbreAux qui pointe vers l'arbre auxiliaires qui contient les positions
  des lettres à utiliser pour obtenir les mots reconnus par ce noeud principal.
\item Un champs visite ayant le même usage que celui de la structure d'arbre auxiliaire.
\item Un champs hache ayant le même usage que celui de la structure d'arbre auxiliaire.
\item Un champs numero ayant le même usage que celui de la structure d'arbre auxiliaire.
\end{itemize}

\subsection{Gestion des arguments}

\noindent Pour la gestion des arguments, on utilise la structure suivante :

\begin{lstlisting}[language=c]
#define NB_ARGUMENT 5

typedef struct argument{
    char options[NB_ARGUMENT]; 
    char* nomFichier;
} Argument;
\end{lstlisting}

Cette structure contient une table de caractères qui seront utilisés comme des booléens
lors du traitement des arguments du programme. Chaque case de ce tableau corresponds à
une option valide pour le programme, cette case contiendra vrai après traitement si 
l'option en question était présente sur la ligne de commande.

\subsection{Lecture du dictionnaire}

\noindent La lecture du dictionnaire se fait grâce à la structure suivante :

\begin{lstlisting}[language=c]
typedef struct dictionnaire {
    FILE* fichier;
    char nom[TAILLE_FICHIER];
    char buffer[TAILLE_BUFFER];
    int curseur;
    int nbcaractere;
    int taillemot;
} Dictionnaire;
\end{lstlisting}

Cette structure de données nous permet de faciliter la lecture des mots dans le fichier
dictionnaire. Elle contient les champs suivant : le fichier texte contenant l'ensemble des mots du 
dictionnaire, le nom du fichier, un buffer qui sert à stocker les caractères du fichier pour
éviter un trop grand nombre d'appels système, un curseur représentant
la position durant le parcours du buffer, un champ nbcaractere qui indique si
le buffer est plein ou non et un champ taillemot qui renseigne sur la
taille du mot alloué dynamiquement pour d'éventuelles réallocations. Nous
décrirons plus en details dans la suite du rapport comment cette
structure est utilisée. 

\subsection{Fusion}

\noindent Voici la structure de liste chainée dont la fusion a besoin :

\begin{lstlisting}[language=c]
typedef struct maillonNoeudPrincipal {
    NoeudPrincipal* noeud;
    struct maillonNoeudPrincipal* suivant;
} MaillonNoeudPrincipal, *ListeNoeudPrincipal;
\end{lstlisting}

La structure de donnée utilisée pour la fusion est un tableau de liste chainée. Ces 
listes contiennent un pointeur sur un noeud principal. Leur usage sera décrit plus tard
dans le rapport.

\subsection{Minimisation}

\noindent Voici les structures de données utilisées pour la minimisation des arbres :

\begin{lstlisting}[language=c]
typedef enum {VIDE, OCCUPE} Etat;

typedef struct caseTable {
    void* adresse;
    unsigned int hache;
    Etat etat;
} CaseTable;

typedef struct table {
    CaseTable* tableau;
    unsigned int tailleTable;
    int (*comparer)(void*, void*);
} Table;
\end{lstlisting}

L'algorithme de minimisation des arbres de ce projet emploie une table
de hachage. Pour implémenter cet algorithme, nous avons choisi
d'utiliser du hachage fermé, par conséquent, il a été necessaire
d'ajouter un champs etat à la structure de case de la table. Ce champs
indique si la case en question est utilisée ou non. Chaque case contient
un pointeur générique, nous avons choisi ce type de pointeur pour
stocker les adresses pour pouvoir stocker différents types de structures
(noeud principaux et auxiliaires dans notre cas). Chaque case connaît
la valeur du hache de l'élément qu'elle contient pour des raisons
pratiques. En effet, il n'est pas possible d'accéder à la valeur du
hache du noeud contenu depuis le pointeur générique.

\bigskip

La structure de la table contient un tableau de case qui représente la
table elle même, un champs tailleTable qui indique la taille de ce 
tableau et un pointeur sur une fonction de comparaison servant à
éliminer les collisions non souhaitées.

\subsection{Sauvegarde des mots dans l'ordre lexicographique}

Voici la structure de donnée utilisée pour la sauvegarde des mots dans l'ordre 
lexicographique :

\begin{lstlisting}[language=c]
typedef struct _ATR
{
    struct _ATR *gauche;
    struct _ATR *milieu;
    struct _ATR *droit;
    char c;
}Noeud, *ATR;
\end{lstlisting}

\subsection{Sauvegarde de l'arbre sous forme de graphe}

Voici la structure de donnée que nous utilisons pour la sauvegarde de l'arbre
sous forme de graphe :

\begin{lstlisting}[language=c]
typedef enum {OUI, NON} Affiche;

typedef struct {
    void* etat;
    int numero;
    Affiche affiche;
} Case;

typedef struct {
    Case *table;
    int indice;
    int taille;
} Tableau;
\end{lstlisting}

Pour nous permettre d'attribuer à chaque noeuds un numero unique nous nous servons d'une table
de hachage.

\subsection{Libération}

\noindent Voici la structure de donnée utilisée pour la libération :

\begin{lstlisting}[language=c]
typedef enum {VIDE, OCCUPE} Etat;

typedef struct caseLiberateur {
    void* adresse;
    Etat etat;
} CaseLiberateur;
\end{lstlisting}

Le fait qu'on minimise les arbres auxiliaires et principaux entraine qu'on mette en place
un système de libération particulier pour ne pas libérer deux fois le même noeud. Pour ce
faire, on utilise une table de hachage dont l'usage sera décrit plus tard dans le 
rapport.

\section{Algorithmes du projet}

\subsection{Fonctions d'initialisation du dictionnaire}

La première étape pour l'initialisation consiste à recupèrer les arguments transmis au programme
à l'aide de notre structure Argument.
Pour initialiser les structures de données necessaire au fonctionnement du dictionnaire T9
il faut recuperer tous les mots contenu dans le fichier dictionnaire transmis en argument
et les inserer dans l'arbre.

\subsection{Recherche normale}

L'algorithme de recherche normale est le suivant : L'utilisateur saisi un mot,
si celui-ci est valide, on descends dans l'arbre principal en suivant les 
branches indiquées par ce mot, s'il est impossible de descendre 
dans l'arbre principal avant qu'on ait lu la totalité du mot, alors aucun mot ne
correspond à son entrée. Une fois qu'on a lu la totalité du mot, on examine le
noeud de l'arbre principal sur lequel on est arrivé. Si ce noeud pointe sur 
un arbre auxiliaire, alors il y
a des mots qui correspondent à l'entrée de l'utilisateur, pour les trouver,
on va parcourir l'arbre auxiliaire en question. On va parcourir toutes les 
branches faisant la même taille que le mot entré par l'utilisateur. Durant le 
parcours de ces branches, on mémorise au fur et à mesure dans un buffer les 
numéros qui correspondent aux branches empruntées. 
Une fois arrivé à la bonne hauteur et si le noeud courant est 
terminal, on affiche le mot qui corresponds à la combinaison de caractères entrés
par l'utilisateur et le contenu du buffer du parcours de l'arbre auxiliaire.

\subsection{Recherche préfixe}

L'algorithme de recherche de mot par préfixe est basé sur la même logique que
l'algorithme de recherche normale. L'utilisateur saisi un mot, si celui-ci est
valide, on cherche le noeud de l'arbre principal qui lui correspond. Une fois
qu'on l'a trouvé, on va parcourir l'arbre principal ayant comme racine ce 
noeud en stockant dans un buffer les numéros correspondant aux branches 
empruntées. Si durant ce parcours, un noeud courant possède un arbre auxiliaire,
on le parcours de la même façon que dans la recherche normale à la différence 
près que les caractères qui vont être affichés sont déterminés dans un premier 
temps par la combinaison du mot entré par l'utilisateur et le contenu du buffer
du parcours de l'arbre auxiliaire jusqu'à ce que la hauteur courante du parcours
de l'arbre auxiliaire soit égale à la taille du mot, puis ensuite l'affichage 
sera fait en fonction du buffer de parcours de l'arbre principal et du buffer de
parcours de l'arbre auxiliaire.

\subsection{Fusion}

Notre algorithme de fusion se base sur le fait que deux arbres auxiliaires 
situés à des hauteurs différentes dans l'arbre principal peuvent toujours être
fusionnés.

\bigskip

L'algorithme est le suivant : On crée un tableau de listes chainées de taille 
équivalente à la hauteur de l'arbre principal. On effectue ensuite un 
parcours en profondeur de l'arbre principal. Durant ce parcours, si un noeud
courant possède un arbre auxiliaire, on stocke l'adresse de ce noeud principal
dans la liste du tableau dont l'indice correspond à la hauteur courante
dans l'arbre principal. Une fois que l'arbre à été entièrement parcouru, on va 
fusionner les arbres. Pour ce faire, on stocke dans un tableau un noeud de 
chaque liste du tableau de liste (tant que c'est possible). Tous les arbres 
de ce tableaux sont forcement fusionnables puisqu'ils sont situés à des 
hauteurs différentes dans l'arbre principal. On fusionne ensuite tous les 
arbres auxiliaires contenus dans les noeuds principaux de ce tableau. On 
continue tant que le tableau de liste contient des noeuds.

\subsection{Minimisation}

\subsubsection{Arbres auxiliaires} 

Pour minimiser les arbres auxiliaires, nous avons employé l'algorithme
qui nous a été donné dans le sujet du projet. Cependant, en application, quelques 
problèmes se posent avec cet algorithme, en effet, dans l'explication du théorème,
la fonction de hachage est supposée parfaite et la table suffisement grande pour
ne pas à avoir a utiliser l'opérateur modulo. Cependant, en pratique, ces deux 
conditions ne sont pas remplies, par conséquent il est possible d'avoir plusieurs
noeuds qui possèdent le même haché alors qu'ils ne doivent pas être fusionnés pour
pallier à ce problème, nous avons choisi d'utiliser une fonction de comparaison
chargée de dire s'il est possible de fusionner deux noeuds. De plus, le fait qu'il
est possible d'obtenir des collisions non souhaitées sur les valeurs des hachés
entraine le fait que cette fonction de comparaison ne peut pas se servir des valeurs
des hachés pour comparer les noeuds. Par conséquent, nous avons du introduire un 
nouveau champs ``numero'' dans la structure d'arbre auxiliaire pour identifier un
noeud qui va être présent après la minimisation de manière unique.

\bigskip

\noindent L'algorithme de comparaison est donc le suivant :

\begin{verbatim}
booléen compare(noeud1, noeud2) {
    Si les statuts des noeuds sont différents :
    |   Retourner faux.

    Pour chaque numero de fils i :
    |   Si le fils i du noeud1 est null et que le fils i du noeud2 n'est pas null ou
    |   Si le fils i du noeud1 n'est pas null et que le fils i du noeud2 est null ou
    |   Si les deux fils ne sont pas null et qu'ils ont des numeros différents :
    |
    |   |   Retourner faux.

    Retourner vrai.
}
\end{verbatim}

\noindent Dans cet algorithme, on ne compare pas la valeur du haché des noeuds 
car cette vérification est faite au niveau de la table de hachage. 

\bigskip

\noindent Voici l'algorithme de la fonction de hachage des noeuds auxiliaires :

\begin{verbatim}
int hacher(arbre, maximum) {
    int resultat = 922337206. /* Valeur arbitraire */
    int tailleEntier = taille des entiers en mot machine (sizeof).
    int tailleMotMachine = taille du mot machine (CHAR_BIT).
    int nombreFils = 0.

    Pour chaque fils de numéro i de arbre :
    |   Si le fils numéro i n'est pas null :
    |   |   tmp = haché du fils numéro i.
    |   |
    |   |   tmp = (tmp >> ((i % tailleEntier) * tailleMotMachine)) |
    |   |         (tmp << ((i % tailleEntier) * tailleMotMachine)).
    |   |   resultat ^= tmp;
    |   |   nbFils++;

    Si nombreFils est pair : 
    |   resultat = (resultat >> ((tailleEntier / 2) * tailleMotMachine)) |
    |              (resultat << ((tailleEntier / 2) * tailleMotMachine)).

    resultat = resultat % maximum;

    Si arbre est final :
    |   resultat = resultat | 1.
    Sinon :
    |   resultat = resultat & ~1.

    Retourner resultat.
}
\end{verbatim}

Cette fonction de hachage permet de répartir de façon convenable les valeurs 
qu'elle renvoie dans l'interval des valeurs autorisées. La totalité de 
l'information disponible est utilisée car aucun dépassement de capacité n'est 
authorisé. Le fait que l'on change la parité du haché selon le statut (terminal 
ou non) du noeud entraine que la taille de la table doit être paire. En effet,
si la taille de la table est pair, la valeur maximale que l'on peut trouver 
grâce au modulo est impair et par conséquent changer la valeur de ce dernier
bit ne peut pas avoir de conséquence facheuses.

\subsubsection{Arbre principal}

Durant la réalisation de ce projet, nous avons eu un problème durant la 
réalisation de la minimisation des arbres principaux. En effet, nous n'avons pas
réussi à mettre en place cette fonctionnalité de manière satisfaisante.

Le problème vient du fait que les liens formés après la minimisation ne sont pas 
corrects, ce qui provoque un gain de mot reconnaissable ainsi qu'une perte de noeuds
auxiliaires. Nous pensons que cette erreur proviens de notre algorithme de 
comparaison de noeud qui est le suivant :

\begin{verbatim}
booléen compare(noeud1, noeud2) {
    Si les statuts ou les profondeurs des noeuds sont différents :
    |   Retourner faux.

    Pour chaque numero de fils i :
    |   Si le fils i du noeud1 est null et que le fils i du noeud2 n'est pas null ou
    |   Si le fils i du noeud1 n'est pas null et que le fils i du noeud2 est null ou
    |   Si les deux fils ne sont pas null et qu'ils ont des numéros différents :
    |
    |   |   Retourner faux.

    Retourner vrai.
}
\end{verbatim}

La fonction de hachage utilisée est du même type que celle des arbres auxiliaires, 
cependant, elle utilise aussi la valeur du haché de la racine de l'arbre auxiliaire
contenu dans le noeud ainsi que la profondeur pour le calcul du haché. 

\bigskip

\noindent En utilisant le dictionnaire dela-fr fourni pour le projet, on 
obtient le résultat suivant :

\begin{verbatim}
Statistiques de l'arbre ...
        Nombre de mots dans l'arbre             : 610631
        Nombre de noeuds de l'arbre principal   : 670097
        Nombre de noeuds des arbres auxiliaires : 574061
\end{verbatim}

\noindent Au lieu des résultats suivant sans minimiser les arbres principaux :

\begin{verbatim}
Statistiques de l'arbre ...
        Nombre de mots dans l'arbre             : 609737
        Nombre de noeuds de l'arbre principal   : 1053622
        Nombre de noeuds des arbres auxiliaires : 654815
\end{verbatim}

Pour tester ces résultats, il est possible de décommenter le code source 
de la fonction optionM correspondant dans le fichier option.c

\subsection{Statistiques}

\subsubsection{Calcul du nombre de mots}

L'algorithme utilisé pour compter le nombre de mots reconnus par le dictionnaire
est le suivant : On parcours l'arbre principal en mémorisant la hauteur 
courante, si durant ce parcours, un noeud principal possède un noeud auxiliaire,
on va parcourir cet arbre auxiliaire pour compter le nombre de noeuds 
terminaux situés à une profondeur équivalente à la profondeur courante de l'arbre
principal. Le nombre de mots reconnus par l'arbre principal est la somme des 
résultats de tous ces parcours.

\bigskip

\noindent Cet algorithme est valable sur les arbres minimisés ou non.

\subsubsection{Calcul du nombre de noeuds des arbres}

Pour calculer le nombre de noeuds d'un arbre principal ou auxiliaire d'un arbre, 
il a été nécessaire 
d'ajouter un champs visite dans leurs structures. Ce champs 
permet d'indiquer si un noeud a déjà été visité ou non pour éviter que le même
noeud soit compté plusieurs fois. Il est nécessaire d'appeler une fonction chargée
de réinitialiser la valeur de ce champs après un parcours durant lequel il a été
utilisé.

\bigskip

\noindent L'algorithme pour l'arbre principal est le suivant : 

\bigskip

\begin{verbatim}
int compteNoeud(arbre) {
    int somme = 1.

    Si arbre est null ou a déjà été visité :
    |   Retourner 0.

    Marquer le noeud comme visité.

    Pour chaque fils f de arbre :
    |   somme = somme + compteNoeud(f).

    Retourner somme.
}
\end{verbatim}

\bigskip

L'algorithme pour les arbres auxiliaires est le suivant : On parcours l'arbre principal, 
pour chaque noeud principal qui possède un arbre auxiliaire, on parcours ce dernier en
comptant le nombre de noeuds et en les marquant de la même façon que dans l'algorithme pour 
compter le nombre de noeuds de l'arbre principal.

Pour simplifier le programme, cette méthode de calcul est utilisée

\subsection{Sauvegarde des mots dans l'ordre lexicographique}

Pour effectuer une sauvegarde complète de tous les mots présents dans l'arbre dans l'ordre
lexicographique, nous avons décidé d'utiliser une structure de données particulière nous 
permettant facilement de ranger les mots dans l'ordre demandé. En effet, nous avons choisit
d'inserer tous les mots contenu dans l'arbre dans un arbre binaire de recherche ou un ATR.
Car il paraissait trop compliqué de pouvoir afficher les mots au fur et à mesure du parcours
de l'arbre. L'algorithme est donc le suivant :

\bigskip

Nous parcourons l'arbre contenant tous les mots du dictionnaire. A chaque nouveau mot trouvé
dans l'arbre, nous l'inserons dans l'ATR. En effet l'ATR va trier le mot pour le ranger au 
bon endroit et dans un ordre lexicographique. Une fois l'ensemble des mots du dictionnaire
inserés dans l'ATR il ne reste plus qu'a affiche l'integralité de son contenu dans le fichier
souhaité. Ainsi les mots sont affichés dans l'ordre lexicographique.

\subsection{Sauvegarde de l'arbre sous forme de graphe}

Pour effectuer la sauvegarde de l'arbre sous forme de graphe, nous avons utiliser une table de hachage
fermé pour affecter chacun des noeuds auxiliaires comme principaux à un numero propre et unique.

\subsection{Libération}

Pour libérer les arbres durant l'exécution ou à la fin du programme, nous avons du
utiliser une structure de donnée qui permet de ne pas libérer plusieurs
fois la même adresse. En effet, après la minimisation, plusieurs noeuds
peuvent faire référence sur la même adresse, par conséquent la phase de
libération est problématique.

\bigskip

Pour résoudre ce problème, nous avons implémenter une table de hachage
dans laquelle on stocke les adresses qu'on souhaite libérer. Si un
adresse est déjà présente et qu'on souhaite l'inserer de nouveau, la
fonction de libération ne fait rien. Après avoir stocker l'ensembles des
adresses que l'on souhaite liberer dans la table de hachage on appelle
une fonction qui parcours l'ensemble de la table de hachage en la vidant
et en libérant les adresses contenues dans toutes les cases pleines
qu'elle rencontre.

\section{Idées d'amélioration}

Pour améliorer notre programme, il est possible d'implémenter le chargement 
d'un graphe depuis un fichier comme celui crée par l'option -b. 

D'un point de vue technique, il peut être interessant de programmer une 
surcouche à la fonction malloc qui permetrais de diminuer de façon très 
importante le nombre d'appel à malloc qui seraient fait tout au long du 
programme. Le problème de cette amélioration est la libération de zones 
mémoires situées entre des zones que l'on souhaite conserver. Pour palier
à ce problème, il serait possible de permettre à l'utilisateur de creer une 
structure lui permettant de faire des ``séries d'allocations'' qui pourraient
être libérées independement les unes des autres. 

\section{Conclusion}

En conclusion de ce projet, nous pouvons dire que nous avons pu voir en détail
l'implémentation des tables de hachages dans d'autre but que le stockage et la
recherche d'élément simple. Nous avons aussi pu apprendre à structurer notre 
code de manière cohérente ainsi qu'à nous conformer à des normes strictes que 
sont -Wall -ansi et -pedantic. Ce projet nous a donner de l'expérience dans les
domaines du choix d'algorithmes et de structures de données, l'implémentation 
d'algorithmes théoriques ainsi que la résolution des problèmes dus aux 
contraintes d'implémentation.

\end{document}
